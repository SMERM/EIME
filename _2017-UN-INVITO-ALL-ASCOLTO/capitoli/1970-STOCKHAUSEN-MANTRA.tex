%-------------------------------------------------------------
%---------------------------- KARLHEINZ STOCKHAUSEN - MANTRA -
%-------------------------------------------------------------

\chapter*{Karlheinz Stockhausen.\\\emph{Mantra}. 1970}
\addcontentsline{toc}{chapter}{Karlheinz Stockhausen. \emph{Mantra}. 1970}

	\begin{flushright}
		\textit{Nella nostra anima c'� una incrinatura che, se sfiorata, \\
		risuona come un vaso prezioso riemerso dalle profondit� della terra} \\
		Wassilly Kandinsky - \emph{Lo Spirituale nell'Arte}
	\end{flushright}

	\begin{flushright}
		\textit{Music of Changes // John ChAnGEs} \\
		Pierre Boulez
	\end{flushright}

	\begin{flushright}
		\textit{Si dice che i compositori abbiano orecchio per la musica e \\
		di solito significa che non sentono nulla che arrivi alle loro orecchie. \\
		Le loro orecchie sono murate dai suoni di loro creazione.} \\
		John Cage - \emph{45' for a Speaker} (1954)
	\end{flushright}

\bigskip

\begin{multicols}{2}
1969 quattro persone che chiacchierano di questo e quello in una macchina tra medicine connecticut e boston.
lungo la strada scrive su una busta che aveva in tasca una melodia che contiene tutte e dodici le note. un anno dopo
inizia il suo lavoro per due pianoforti e riprende questa melodia.
tutta la melodia stirata sull'intera durata del brano, un'ora. e contemporanemamente compressa nella pi� piccola portzione temporale.
ogni nota a sua volta richiamava a se tutte le altre note rendendo per ognuno di questi punti il complesso delle 12 note. Il tutto somiglia
molto ad un sistema di stelle.

tutta la melodia � il mantra, come fosse una formula. ci sono 4 regioni separate da pause. la prima regione � formata da 4 note. la seconda da 2. la terza da 5 la quarta da 3

mirror

spiegher� come il mantra, la formula, pu� essere usata per l'intera composizione, per fare questo ho bisogno della variazione

trovare qualcosa sulle lezioni inglesi?

dobbiamo immaginare come se ogni nota determina un'intera sezione di una certa composizione

combinare le sezioni tra loro con la tecnica delle variazioni

non ho costruito la formula sulla base di una scala cromatica 

\end{multicols}


\begin{table}[htp]
\caption{Melodia - Formula - Regioni}
\begin{center}
\begin{sf}{\footnotesize
\begin{tabular}{r c c c c c c c c c c c c c c c c c c }

       \textbf{altezze} & \multicolumn{4}{c}{4} 	 & \multirow{3}*{pausa 3} & \multicolumn{3}{c}{2} & \multirow{3}*{pausa 2} & \multicolumn{5}{c}{5}  & \multirow{3}*{pausa 1} & \multicolumn{3}{c}{3}  \\
\textbf{durata regione} & \multicolumn{4}{c}{10} &                        & \multicolumn{3}{c}{6} &                        & \multicolumn{5}{c}{15} &                        & \multicolumn{3}{c}{12}   \\
  \textbf{suddivisione} & 1 & 2 & 3 & 4          &                        & 2 & 2 & 2             &                        & 5 & 2 & 1 & 3 & 4      &                        & 4 & 2 & 6 \\

\end{tabular}}
\end{sf}
\end{center}
\label{default}
\end{table}%
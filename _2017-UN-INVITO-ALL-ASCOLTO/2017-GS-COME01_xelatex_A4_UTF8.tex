%!TEX TS-program = xelatex
%!TEX encoding = UTF-8 Unicode

\documentclass[%10pt,
			   a4paper,
			   twoside
			   ]{book}
\usepackage[margin=1in]{geometry}

\usepackage[parfill]{parskip}    % Activate to begin paragraphs with an empty line rather than an indent
\usepackage{graphicx}
\usepackage{amssymb}
\usepackage{wrapfig}
\usepackage{float} % they need to be placed in specific locations with the [H] (e.g. \begin{table}[H])

\usepackage{paralist}
\usepackage{array}
\usepackage{longtable}
\usepackage{multirow}

\usepackage{url}

\usepackage{multicol}
\usepackage{pdfpages}

%-------------------------------------------------------------
%----------------------------------------------------- FONTS -
%-------------------------------------------------------------

\linespread{1.04}

\usepackage{etoolbox}

\usepackage{fontspec,
			xltxtra,
			xunicode
			}

\defaultfontfeatures{Mapping=tex-text}

\setromanfont[Mapping=tex-text]{Alegreya}

\setsansfont[Scale=MatchLowercase,
			 Mapping=tex-text
			 ]{Fira Sans}

\setmonofont[]{Fira Mono}

\newfontfamily{\scaps}{Alegreya SC}

\newfontfamily\quotefont{Alegreya}
\AtBeginEnvironment{quote}{\quotefont\small}

\usepackage[italian,
			english
			]{babel}

\usepackage{lilyglyphs}

\usepackage[hang,
			small,
			labelfont=bf,
			up,
			textfont=it,
			up
			]{caption}

%-------------------------------------------------------------
%---------------------------------------------------- HEADER -
%-------------------------------------------------------------

\usepackage{fancyhdr}

\fancypagestyle{plain}{%
\fancyhf{} % clear all header and footer fields
%\fancyhead[CO,CE]{---Draft---}
\fancyfoot[RO,LE]{\fontsize{18pt}{18pt}\selectfont\thepage} % except the center
\renewcommand{\headrulewidth}{0pt}
\renewcommand{\footrulewidth}{0pt}}
\pagestyle{plain}

%-------------------------------------------------------------
%----------------------------------------------------- TITLE -
%-------------------------------------------------------------

\makeatletter
%\frameattext{<backgroundcolor>}{linecolor}{<linewidth>}
\newdimen\extraxsep
\newdimen\extraysep
\extraxsep=20mm
\extraysep=20mm
\newcommand\frameattext[3]{%
  \linethickness{#3}%
  \AddToShipoutPicture*{%
    \AtTextLowerLeft{%text-boder
       \put(\LenToUnit{-,5\extraxsep},\LenToUnit{-0.5\extraysep}){\color{#1}%
             \rule{\dimexpr\textwidth+\extraxsep\relax}{\dimexpr\textheight+\extraysep\relax}}%
       \put(\LenToUnit{-,5\extraxsep},\LenToUnit{-0.5\extraysep}){\color{#2}%
       \framebox(\LenToUnit{\dimexpr\textwidth+\extraxsep\relax},%
                 \LenToUnit{\dimexpr\textheight+\extraysep\relax}){}
       }
    }%
  }%
}
%\frameatpage{<backgroundcolor>}{linecolor}{<linewidth>}
\newcommand\frameatpage[3]{%
  \linethickness{#3}%
  \AddToShipoutPicture*{%
    \AtPageLowerLeft{%page-border
      \put(0,0){\color{#1}\rule{\paperwidth}{\paperheight}}%
      \put(\LenToUnit{\@wholewidth},\LenToUnit{\@wholewidth}){%
       \color{#2}\framebox(\LenToUnit{\dimexpr\paperwidth-2\@wholewidth\relax},%
                  \LenToUnit{\dimexpr\paperheight-2\@wholewidth\relax}){}%
      }%
    }%
  }%
}

\makeatother

%-------------------------------------------------------------
%-------------------------------------------------- DOCUMENT -
%-------------------------------------------------------------

\begin{document}

\setlength{\columnsep}{.4in}

\frameattext{white}{black}{2pt}

\begin{center}
	~\\
	\vfill
    \fontsize{19}{19}\selectfont{Giuseppe SILVI} \\
		\vspace{1cm}
    \fontsize{81}{54}\selectfont{EIME \#01} \\
	\vfill
	\fontsize{19}{10}\selectfont{\emph{un invito all'ascolto}} \\
		\vspace{1cm}
	\fontsize{19}{19}\selectfont{draft 001\\
								 \today}
	\vfill

\end{center}
%\maketitle

\thispagestyle{empty}

%\clearpage
%
%\thispagestyle{empty}
%
%~
%
%\clearpage
%
%\tableofcontents
%
%~\vfill
%
%\includegraphics[width=.25\columnwidth]{images/by-nc-sa}\\
%\emph{A. Sax.} by Giuseppe Silvi is licensed under a Creative Commons \\
%Attribution-NonCommercial-ShareAlike 4.0 International License.\\
%Permissions beyond the scope of this license may be available at\\
%giuseppesilvi.com/asax.%\marginpar{prova}
%
%
\clearpage

%-------------------------------------------------------------
%----------------------------------------------- DESCRIPTION -
%-------------------------------------------------------------

%\chapter*{Abstract}
%\noindent\makebox[\linewidth]{\rule{\paperwidth}{9pt}}
%\addcontentsline{toc}{chapter}{Abstract}

\chapter*{John Cage. \emph{Cartridge Music}.}

	\begin{flushright}
		\textit{Nella nostra anima c'è una incrinatura che, se sfiorata, \\
		risuona come un vaso prezioso riemerso dalle profondità della terra} \\
		Wassilly Kandinsky - \emph{Lo Spirituale nell'Arte}
	\end{flushright}

	\begin{flushright}
		\textit{Music of Changes // John ChAnGEs} \\
		Pierre Boulez
	\end{flushright}

	\begin{flushright}
		\textit{Si dice che i compositori abbiano orecchio per la musica e \\
		di solito significa che non sentono nulla che arrivi alle loro orecchie. \\
		Le loro orecchie sono murate dai suoni di loro creazione.} \\
		John Cage - \emph{45' for a Speaker} (1954)
	\end{flushright}

\bigskip

\begin{multicols}{2}

\begin{quote}
	Ogni opera d'arte è figlia del suo tempo, e spesso è madre dei nostri
	sentimenti.

Analogamente, ogni periodo culturale esprime una sua arte, che non si ripeterà
mai più.
Lo stesso sforzo di ridar vita a principi estetici del passato può creare al
massimo delle opere d'arte che sembrano bambini nati morti.
Noi non possiamo, ad esempio, avere la sensibilità e la vita
interiore\footnote{Wassilly Kandinsky, \emph{Lo Spirituale nell'Arte},
SE. 1989}\ldots
\end{quote}

di John Cage, in luogo degli antichi Greci, come avrebbe continuato Kandinsky.
Ci deve essere un certo grado di consapevolezza in relazione al livello di
comprensione-incomprensione del pensiero di John Cage. Ma ammettendo di
averlo compreso, per quanto noi potremmo approfondire lo studio del suo
pensiero e della sua musica, potremmo solo arrivare ad imitarne alcuni tratti
stilistici. E se tentassimo di

\begin{quote}
adottare i loro princìpi, non faremmo che produrre forme simili alle loro,
ma prive di anima\footnote{\emph{idem}}.
\end{quote}

Questo non esclude che si possa riuscire ad entrare in contatto con le
motivazioni e gli stimoli artistici, soprattutto se si attinge ai tratti
condivisi tra le somiglianze delle forme artistiche,

\begin{quote}
delle aspirazioni interiori e degli ideali (che un tempo erano stati raggiunti
e poi vennero dimenticati), la somiglianza cioè fra i climi culturali di due
epoche che può portare alla ripresa di forme che erano già state utilizzate in
passato per esprimere le stesse tensioni\footnote{\emph{idem}}.
\end{quote}

Tra le caratteristiche non convenzionali di John Cage ce n'è per me una piuttosto
curiosa: davanti ad una mastodontica produzione musicale, ad una quasi totale
assenza di apparato critico strutturato, non è difficile individuare gli
\emph{stili musicali} di John Cage, dividendoli in quattro sezioni temporali
scandite da date piuttosto precise: 1939, 1951, 1969.

Le prime opere sono caratterizzate da un cromatismo strutturato e dalla
sperimentazione soprattutto con gli strumenti a percussione.
Con \emph{Firts Construction in Metal} del 1939 realizza la prima opera
utilizzando strutture temporali. Nel 1951 con \emph{Music of Changes} e
\emph{Imaginary Landscape No. 4} si compie il passaggio dall'organizzazione
sistematica alla combinazione di elementi sistematici, gusti personali e
indeterminazione casuale. Gli anni che seguirono il 1951 furono decisivi per
lo sviluppo delle operazioni casuali.

È dal 1957 che Cage inizò a concepire opere nelle quali tutti gli aspetti
dell'interpretazione fossero indeterminati. Tutte le decisioni in merito ai
suoni e alla loro successione sono delegate dal compositore all'esecutore;
la partitura permette soltanto di assicurare una certa disciplina quanto al
modo di porendere devisioni che produrranno dei risultati imprevedibili. In quegli
anni \emph{Fontana Mix, Cartridge Music}\footnote{\emph{Cartridge
Music} at John Cage's Database of Works www.johncage.org: \\ This work was later used as music for the choreographed piece by Merce Cunningham
entitled Changing Steps, with stage and costume design by Charles Atlas (from 1973,
Mark Lancaster); still later, it was used for the choreographed pieces by
Cunningham entitled Exercise Piece II and Exercise Piece III.
The word 'Cartridge' in the title refers to the cartridge of phonographic
pick-ups, into the aperture of which is fitted a needle. In Cartridge Music,
the performer is instructed to insert all manner of unspecified small objects
into the cartridge; prior performances have involved such items as pipe cleaners,
matches, feathers, wires, etc. Furniture may be used as well, amplified via
contact microphones. All sounds are to be amplified and are controlled by the
performer(s). The number of performers should be at least that of the cartridges,
but not greater than twice the number of cartridges. Each performer makes his or
her own part from the materials provided: 20 numbered sheets with irregular
shapes (the number of shapes corresponding to the number of the sheet) and 4
transparencies, one with points, one with circles, another with a circle marked
like a stopwatch, and the last with a dotted curving line, with a circle at one
end. These transparencies are to be superimposed on one of the 20 sheets, in
order to create a constellation from which one creates one's part. It is also
possible to create other pieces from these materials, such as Duet for Cymbal
or a Piano Duet. Cage also used Cartridge Music as a means to compose several
of his lectures, including “Where Are We Going? And What Are We Doing?” (1960),
“Rhythm, Etc.” (1962), “Jasper Johns: Stories and Ideas” (1963), and
“On Robert Rauschenberg, Artist, and His Work” (1961).} e la serie delle \emph{Variations},
consisteva in fogli lucidi trasparenti, con linee, punti e curve, a partire dalle
quali si costruivano partiture (strutture, materiali e relazioni, applicabili non
solo alla musica) che era possibile eseguire in diverse circostanze.

La regolarità e la continuità estetica di questa produzione fu interrotta nel 1969
con il brano \emph{HPSCHD} per sette clavicembali amplificati e tape multicanale.
La musica di Cage ri riappropria quindi di una notazione convenzionale in uninione
con la tecnicna del \emph{collage}. L'anno successivo Cage porse le prime domande
di composizione all'\emph{I Ching}.

\begin{quote}
	I procedimenti casuali sono solo uno strumento tra altri che Cage ha utilizzato
	per perseguire coerentemente un unico scopo: l'accettazione disciplinata, in
	contesti musicali, di ciò che fino ad allora era stato rifiutato. «Sono sempre
	stato dalla parte delle cose che non si devono fare», ha osservato una volta,
	«cercando il modo di rimettere in gioco gli elementi rifiutati»\footnote{William
	Brooks, \emph{Scelte e Cambiamenti Nella Musica Recente di Cage} 1982}.
\end{quote}

\centering{***}

\begin{quote}
	L'artista cercherà di suscitare sentimenti più delicati, senza nome [\ldots]
	Attualmente però lo spettatore è quasi sempre incapace di emozioni.
	Nell'opera d'arte cerca una mera imitazione della natura a scopo pratico.
	[\ldots] certo l'immedesimazione (e la contrapposizione) non deve essere
	vacua o superficiale: anzi, l'atmosfera dell'opera deve rendere più
	coinvolgente e visionaria l'atmosfera in cui è immerso lo spettatore. [\ldots]
	L'affinamento e la diffusione della loro voce nel tempo e nello spazio
	rimangono però un fatto soggettivo, che non esaurisce le potenzialità dell'arte.
\end{quote}

\end{multicols}

%\begin{figure}[htbp]
%\begin{center}
%\includegraphics[width=.99\textwidth]{../../../Photos/090415_capolona.JPG}
%\caption*{S.T.ONE \& S.T.ON3S - Centro Ricerche Musicali, Rome - July 2015, 17th}
%\label{default}
%\end{center}
%\end{figure}

\end{document}
